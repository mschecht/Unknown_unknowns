\newpage
\chapter{Abstract}
\chaptermark{}
\renewcommand{\chaptermark}[1]{\markboth{#1}{}}
\renewcommand{\sectionmark}[1]{\markright{#1}}
\fancyhead[RE]{\small}
% Section in the left on even pages}
\fancyhead[LO]{\small\rightmark}%Section in the left on odd pages

The era of environmental metagenomic sequencing has added terabytes of sequencing data to databases. To explore this immense diversity to genes, clustering techniques have been used to group protein coding sequences into protein families (clusters). This has lead to the discovery of novel families and redefined protein diversity. Unfortunately, due to the lack of isolated microbial genomes from the environment and incorrect protein annotation management in databases, many protein families are left without a known function. This constrains microbial functional research and often the \quotes{unknowns} are discounted during microbial functional profiling. There have been many approaches to expand the annotations of known proteins to the unknowns, but still a large proportion of protein families remain uncharacterized. Additionally, there have been attempts to categorize and classify the unknown, but sufficient categories have not been described to include all unknown environmental protein coding sequences. The Vanni \textit{et. al.} (in prep) workflow takes a novel approach to protein clustering, focusing on cluster quality and categorization of the unknown clusters. In this thesis, the resulting clusters from Vanni \textit{et. al.} workflow were used to investigate the distribution and biogeography of the unknown in the TARA Oceans dataset. This investigation demonstrates that when utilizing the whole metagenomic sample (knowns and unknowns) there is increased variation in ordinations which leads to clearer sample site separation. It is also found that the unknown fraction of protein clusters have distinct characteristics of being adaptive proteins. Finally, evidence is presented of an ubiquitous unknown fraction of protein clusters throughout the world's oceans.