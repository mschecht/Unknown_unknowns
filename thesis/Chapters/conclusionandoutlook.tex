% Conclusion and Outlook

% Chapter Template

\chapter{Conclusion and Outlook} % Main chapter title

\label{Conclusion and Outlook} % Change X to a consecutive number; for referencing this chapter elsewhere, use \ref{ChapterX}

%\lhead{Chapter 1. \emph{Introduction}} % Change X to a consecutive number; this is for the header on each page - perhaps a shortened title

\renewcommand{\chaptermark}[1]{\markboth{#1}{}}
\renewcommand{\sectionmark}[1]{\markright{#1}}
\fancyhead[RE]{\small\leftmark}
% Section in the left on even pages}
\fancyhead[LO]{\small\rightmark}%Section in the left on odd pages

%\section{A section goes here}

In this thesis, we have shown that including the entire metagenomic sample in functional biogeography is key to resolve difference between sampling sites. Regardless if a protein cluster has an annotation, it can be accounted for in samples and help define minute differences between geographical sites. By defining a site by their complete functional repertoire, unique insights were generated about the TARA Ocean Southern Ocean sampling sites not observed in the original TARA analysis \citep{Sunagawa_2015}. The Vanni et. al. 2018 clusters lay the groundwork for accounting for the entire metagenomic sample and has many future applications in functional microbiology research.\\

\textbf{Unknowns as indicators}\\


One direction for immediate application could be to include the Unknowns in objectives for genomic observatories. Since EUs and GUs were associated with small niche occupancy and are selected due to specific environmental factors, their abundances may be good indicators for environmental pollution. Microbial populations respond fast to environmental changes and thus are good indicators for ecosystem health and stress \citep{Buttigieg_2018}. Genomic observatories monitor this phenomena to determine ecosystem health \citep{ Davies_2012}. With the immense influx of high-throughput next-generation sequencing (NGS) data, genomic observatories can effectively monitor ecosystem changes using predictive strategies. \cite{ Buttigieg_2018} makes the case that a deeper understanding of ocean microbial interactions and functions will improve biomonitoring. It was recently shown that machine learning approaches can effectively analyze differences in ocean microbial taxa can effectively predict anthropogenic impacts \citep{Cordier_2017}. A future direction for the Unknowns could be to train Random Forest (RF) algorithms to predict the anthropogenic impacts on the environment and find reliable indicators for environmental monitoring.
Including all predicted ORFs in a sample not only adds information to the site, but also is a better return on investment of the sequencing effort itself. Millions taxpayer dollars and euros have been spent on large environmental sequencing efforts. It is a waste of resources that in most cases not all information for the DNA samples is gleaned and the unannotated fraction is discarded.\\

A general theme in this thesis was the application of traditional ecological methods to analyze the distribution of the Unknown fraction. Through Levin\textquotesingle s Niche Breadth and biogeographical analysis we were able to show unique insights about unknown ubiquitous functions in the ocean. Multidisciplinary approaches may be the key to uncovering more insights about functional biogeography in the world's oceans. There is already dialogue discussing how to apply traditional population genetics methods to the core and accessory genes of the pangenome. Another future direction for the research into the Unknown fraction is to examine their distribution in pangenomes. Research questions can be addressed such as are the Unknowns enriched in the core genome, are Unknowns associated with different phylotypes?  \\

\textbf{Revealing the function of unknowns}\\

If 99\% of microbes are currently uncultivable in the laboratory \citep{Barer_2015}, it can be assumed that many functions will remain unknown until more are isolated and cultured. Computational approaches to marine microbiology have their limitations. Regardless if sequencing technology continues to increase in sequencing depth and quality, gene prediction, assembly, and binning algorithms will never be perfect. NGS data will always be subjected to sequencing and post-processing artifacts. Bioinformatics provides a great platform for hypothesis generation to select and search for important Unknowns, but to truly characterize the Unknown, laboratory experiments on cultured isolates are an absolute necessity. If the Ubiquitous EU component signal is indeed coming from the rare biosphere many innovations will be needed to create cutting edge culturing methods to grow these bugs in the laboratory. The potential of new microbial functions in the rare biosphere is immense and worth investigating due to the untapped resources for biotechnology, environmental applications, and fundamental knowledge. \\

Another step to investigate the ubiquitous EU would be to express them in microbial vectors and test their function. Many methods are developing to efficiently screen metagenomic libraries, select contigs, and express them in vectors \citep{Leis_2013}. Once expressed, novel genes can be quickly tested for their functions in assays.\\

Similar to how \cite{Wyman_2017}  created a list of most wanted FUnkFams, this ecological analysis of the Unknown component fraction is a starting point for targeted environmental protein analysis. Analyzing the patterns and distribution of the Unknown from an ecological perspective can lead to more insights about microbial functions. Additionally, the more functions that are uncovered, the more predictive power genomic observatories may have to detect minute changes in the environment.\\

It has been suggested that genes of hypothetical function that are conserved throughout phyla should be prioritized for characterization\citep{Galperin_2004, Hanson_2010}. The \quotes{core} set of genes have the potential to uncover new phylogenetic markers and essential functions for the definition of life on this planet. This philosophy should be extended to environmentally conserved protein clusters, in particular the Environmental Unknown Ubiquitous clusters.\\
