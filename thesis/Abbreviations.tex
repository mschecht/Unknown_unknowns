\newpage
\clearpage % Start a new page
\thispagestyle{plain}
\section*{Abbreviations}
\setstretch{1} % Set the line spacing to 1.5, this makes the following tables easier to read

%\lhead{\emph{Abbreviations}} % Set the left side page header to "Abbreviations"

\begin{acronym}
%\acro{ABC}{ATP-binding cassette}
%\acro{ANI}{average nucleotide identity}
\acro{DUF}{Domain of Unknown Function}
\acro{EU}{Environmental Unknowns}
\acro{FAIR}{Findable Accessible Interoperable Reproducible}
\acro{GU}{Genomic Unknowns}
\acro{HMM}{Hidden Markov Model}
\acro{Kwp}{Knowns without PFAM annotation}
\acro{ORF}{Open Reading Frame}
\acro{OTU}{Operational Taxonomic Unit}
\acro{PCA}{Principal Component Analysis}
\acro{TP}{TARA Ocean prokaryotic metagenomes (all depths)}
\acro{TPS}{TARA Ocean prokaryotic, surface metagenomes}
\end{acronym}

\cleardoublepage{}
\clearpage % Start a new page
\thispagestyle{plain}
\section*{Glossary}
\setstretch{1} % Set the line spacing to 1.5, this makes the following tables easier to read

%\lhead{\emph{Abbreviations}} % Set the left side page header to "Abbreviations"

\begin{acronym}
\acro{Cluster}{A group of protein coding sequences that were grouped together by an unsupervised sequence clustering algorithm.}
\acro{Component}{A group of clusters from the Vanni et. al. workflow that were aggregated together due to PFAM domain architecture redundancy.}
\acro{Environmental Unknowns}{ORF clusters with an unknown function and are not found in sequenced or draft- genomes.}
\acro{Genomic Unknowns}{ORF clusters that have an unknown function (e.g. DUF, hypothetical protein) but are found in sequenced or draft-genomes.}
\acro{Knowns}{ORF clusters that have been annotated with a PFAM domains of known function.}
\acro{Knowns without PFAMs}{ORF clusters that have a known function but do not contain PFAM annotations.}
\end{acronym}
